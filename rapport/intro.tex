% Blabla Introduction

\section{Context} % Expliquer dans quel contexte s'inscrit le stage

    Technological advances in Artificial Intelligence and Machine Learning (MLOps) have led to the development of many tools to facilitate the life of users and methods to advance research. A very recent phenomenon in the world of research, Machine Learning allows to save a lot of computing time to perform large scale simulations but also to make some predictions. 
    
    There are several branches of Machine Learning. The most classical one is the one where a program is given a lot of data and it learns from this data. The program can then provide a prediction based on the input parameters. However, by doing this, we lose the physical sense (if the data are physical simulations or experiments). Another branch is to sort the input data according to what is more likely to happen. The physical meaning is then preserved but the program will only be able to provide an estmiation of what could happen. 

\section{Internship's Problematic}

    This internship is part of a multiscale analysis of fracture and more precisely by conducting a diaglogue in molecular dynamics and description in continuous medium. More precisely, the aim is to identify a cohesive zone model, representing the fracture mechanism at the continuous scale through a "stress-vector" - "opening" relation. This model will be identifiable following calculations in molecular dynamics which produce the numerical experiments. \medskip

    An important part of the work is to conduct Molecular Dynamics simulations on a Silicon (Si) crystal for which the fracture occurs by cleavage. However, since Si has anisotropic elastic properties, it is expected that the fracture properties are also anisotropic. Therefore, simulations for different orientations between the crack plane and the crystalline symmetry planes will also have to be carried out. A systematic approach can be conducted. Nevertheless, the methodology associating Machine Learning and Molecular Dynamics is to be exploited in order to gain in calculation time. \medskip
    
    Once the cohesive model is identified, it is then possible to study and predict the intercations between cracking and microstructure (in a polycrystal for example), as well as between crack and cavity. \medskip

    It is a 100\% digital project with a strong interest in simulation methods and Machine Learning.

\section{Laboratory Presentation}

    \subsection{History}

        The Laboratory of Science and Engineering of Materials and Processes is the result of the merger of three units on January 1, 2007. It is a joint research unit: CNRS, Grenoble-INP, and IESA. It brings together an average of 220 people including 56 researchers and teacher-researchers, 37 engineers, technicians and administrative staff, 60 PhD students, post-doctoral fellows, guests and trainees. 

    \subsection{Groupes de Recherche}

        Le Laboratoire s'appuie sur quatre groupe de recherche qui pérennisent les sciences de base en physique et physico-chimie, thermodynamique et cinétique, mécanique des solides et des fluides : 

        \begin{enumerate}[\hspace{3em}$\bullet$]
            \item \textbf{EPM : }Élaboration par Procédés Magnétiques
            \item \textbf{GPM2 : }Génie Physique et Mécanique des Matériaux
            \item \textbf{PM : }Physique du Métal
            \item \textbf{TOP : }Thermodynamique, modélisation, Optimisation des Procédés
        \end{enumerate}
    
        Ce stage se place entre deux divisions (PM et TOP) dans une petite équipe constituée de : 

        \begin{enumerate}[\hspace{3em}$\bullet$]
            \item \textbf{Noel JAKSE : }Enseignant-Chercheur au groupe de recherche TOP, Maître et Tuteur de Stage
            \item \textbf{Rafael ESTEVEZ : }Chercheur, Co-tuteur de Stage
            \item \textbf{Thibault MROZ : }Stagiaire Assistant Ingénieur
        \end{enumerate}

        Le Groupe de Recherche TOP se concentre sur l'élaboration des matériaux, les phéno-
        mènes thermodynamiques (stabilité et charactérisation) et la modéliation atomistique, thermo-
        dynamique, cinétique et des réacteurs. Cela a des applications dans les domaines des films minces, des alliages métalliques complexes et des matériaux fonctionnels.

        Le Groupe de Recherche PM se concentre sur la métallurgie des métaux : structure atomique, propriétés mécaniques et physiques ainsi que l'oxydation. Cela a des applications dans les domaines des matériaux pour l'énergie et pour la micro-électronique mais aussi pour les matériaux structuraux. 

        Le stage s'inscrit dans le domaine de la modélisation atomistique et de la structure atomique. 

\section{Plan du rapport}
        
