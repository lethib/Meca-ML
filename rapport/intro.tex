% Blabla introduction 
\section{Contexte et Problématique du Stage} % Expliquer dans quel contexte s'inscrit le stage

    Ce stage s'inscrit dans le cadre d'une analyse multi-échelles de la rupture et plus précisément en menant un diaglogue en dynamique moléculaire et description en milieu continu. Plus précisément, il s'agit d'identifier un modèle de zone cohésive, représentant à l'échelle continu le mécanisme de rupture au travers d'une relation "vecteur-contrainte" - "ouverture". Ce modèle sera identifiable suite à des calculs en dynamique moléculaire qui produisent les expérimentations numériques. \\

    Une partie importante du travail est de mener des simulations de Dynamique Moléculaire sur un cristal de Silicium (Si) pour lequel la rupture a lieu par clivage. Cependant, le Si possédant des propriétés élastiques anisotropes, il est attendu que les propriétés de rupture le soient également. Dès lors, des simulations pour différentes orientations entre plan de la fissure / plans de symétrie critallins seront également à mener. Une approche systématique peut être menée. Néanmoins, la méthodologie associant Machine Learning et Dynamique Moléculaire est à exploiter afin de gagner en temps de calculs. \\
    
    Une fois que le modèle cohésif est identifié, il est ensuite possible d'étudier et de prédire les intercations entre fissuration et microstructure (dans un polycristal par exemple), ainsi qu'entre fissure et cavité. \\

    Il s'agit d'un projet "100\% numérique" et porte un fort intérêt pour les méthodes de simulations ainsi que le Machine Learning.