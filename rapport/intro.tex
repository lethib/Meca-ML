% Blabla introduction 
\section{Contexte et Problématique du Stage} % Expliquer dans quel contexte s'inscrit le stage

    Ce stage s'inscrit dans le cadre d'une analyse multi-échelles de la rupture et plus précisément en menant un diaglogue en dynamique moléculaire et description en milieu continu. Plus précisément, il s'agit d'identifier un modèle de zone cohésive, représentant à l'échelle continu le mécanisme de rupture au travers d'une relation "vecteur-contrainte" - "ouverture". Ce modèle sera identifiable suite à des calculs en dynamique moléculaire qui produisent les expérimentations numériques. \\

    Une partie importante du travail est de mener des simulations de Dynamique Moléculaire sur un cristal de Silicium (Si) pour lequel la rupture a lieu par clivage. Cependant, le Si possédant des propriétés élastiques anisotropes, il est attendu que les propriétés de rupture le soient également. Dès lors, des simulations pour différentes orientations entre plan de la fissure / plans de symétrie critallins seront également à mener. Une approche systématique peut être menée. Néanmoins, la méthodologie associant Machine Learning et Dynamique Moléculaire est à exploiter afin de gagner en temps de calculs. \\
    
    Une fois que le modèle cohésif est identifié, il est ensuite possible d'étudier et de prédire les intercations entre fissuration et microstructure (dans un polycristal par exemple), ainsi qu'entre fissure et cavité. \\

    Il s'agit d'un projet "100\% numérique" et porte un fort intérêt pour les méthodes de simulations ainsi que le Machine Learning.

\section{Présentation du Laboratoire}

    \subsection{Historique}

        Le Laboratoire de Science et Ingénierie des Matériaux et Procédés résulte de la fusion de trois unités au 1er Janvier 2007. C'est une UMR, Unité Mixte de Recherche : CNRS, Grenoble-INP, et IESA. Il rassemble en moyenne 220 personnes dont 56 chercheurs et enseignants-chercheurs, 37 ingénieurs, techniciens et administratifs, 60 doctorants, les post-doctorants, invités et stagiaires. 

    \subsection{Groupes de Recherche}

        Le Laboratoire s'appuie sur quatre groupe de recherche qui pérennisent les sciences de base en physique et physico-chimie, thermodynamique et cinétique, mécanique des solides et des fluides : 

        \begin{enumerate}[\hspace{3em}$\bullet$]
            \item \textbf{EPM : }Élaboration par Procédés Magnétiques
            \item \textbf{GPM2 : }Génie Physique et Mécanique des Matériaux
            \item \textbf{PM : }Physique du Métal
            \item \textbf{TOP : }Thermodynamique, modélisation, Optimisation des Procédés
        \end{enumerate}
    
        Pour mon stage j'ai travaillé entre pour deux divisions (PM et TOP) dans une petite équipe constituée de : 

        \begin{enumerate}[\hspace{3em}$\bullet$]
            \item \textbf{Noel JAKSE : }Enseignant-Chercheur au groupe de recherche TOP, Maître et Tuteur de Stage
            \item \textbf{Rafael ESTEVEZ : }Chercheur, Co-tuteur de Stage
            \item \textbf{Thibault MROZ : }Stagiaire Assistant Ingénieur
        \end{enumerate}

        Le Groupe de Recherche TOP se concentre sur l'élaboration des matériaux, les phéno-
        mènes thermodynamiques (stabilité et charactérisation) et la modéliation atomistique, thermo-
        dynamique, cinétique et des réacteurs. Cela a des applications dans les domaines des films minces, des alliages métalliques complexes et des matériaux fonctionnels.

        Le Groupe de Recherche PM se concentre sur la métallurgie des métaux : structure atomique, propriétés mécaniques et physiques ainsi que l'oxydation. Cela a des applications dans les domaines des matériaux pour l'énergie et pour la micro-électronique mais aussi pour les matériaux structuraux. 

        Mon stage s'inscrit dans le domaine de la modélisation atomistique et de la structure atomique. 

    \subsection{Présentation et Déroulement du Stage}