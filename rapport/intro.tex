% Blabla Introduction

\section{Context} % Expliquer dans quel contexte s'inscrit le stage

    Technological advances in Artificial Intelligence and Machine Learning (MLOps) have led to the development of many tools to facilitate the life of users and methods to advance research. A very recent phenomenon in the world of research, Machine Learning allows to save a lot of computing time to perform large scale simulations but also to make some predictions. 
    
    There are several branches of Machine Learning. The most classical one is the one where a program is given a lot of data and it learns from this data. The program can then provide a prediction based on the input parameters. However, by doing this, we lose the physical sense (if the data are physical simulations or experiments). Another branch is to sort the input data according to what is more likely to happen. The physical meaning is then preserved but the program will only be able to provide an estmiation of what could happen. 

\section{Internship's Problematic}

    This internship is part of a multiscale analysis of fracture and more precisely by conducting a diaglogue in molecular dynamics and description in continuous medium. More precisely, the aim is to identify a cohesive zone model, representing the fracture mechanism at the continuous scale through a "stress-vector" - "opening" relation. This model will be identifiable following calculations in molecular dynamics which produce the numerical experiments. \medskip

    An important part of the work is to conduct Molecular Dynamics simulations on a Silicon (Si) crystal for which the fracture occurs by cleavage. However, since Si has anisotropic elastic properties, it is expected that the fracture properties are also anisotropic. Therefore, simulations for different orientations between the crack plane and the crystalline symmetry planes will also have to be carried out. A systematic approach can be conducted. Nevertheless, the methodology associating Machine Learning and Molecular Dynamics is to be exploited in order to gain in calculation time. \medskip
    
    Once the cohesive model is identified, it is then possible to study and predict the intercations between cracking and microstructure (in a polycrystal for example), as well as between crack and cavity. \medskip

    It is a 100\% digital project with a strong interest in simulation methods and Machine Learning.

\section{Laboratory Presentation}

    \subsection{History}

        The Laboratory of Science and Engineering of Materials and Processes is the result of the merger of three units on January 1, 2007. It is a joint research unit: CNRS, Grenoble-INP, and IESA. It brings together an average of 220 people including 56 researchers and teacher-researchers, 37 engineers, technicians and administrative staff, 60 PhD students, post-doctoral fellows, guests and trainees. 

    \subsection{Research Groups}

        The Laboratory relies on four research groups that perpetuate the basic sciences in physics and physical chemistry, thermodynamics and kinetics, solid and fluid mechanics: 

        \begin{enumerate}[\hspace{3em}$\bullet$]
            \item \textbf{EPM : }Elaboration by Magnetic Processes
            \item \textbf{GPM2 : }Physical and Mechanical Engineering of Materials
            \item \textbf{PM : }Metal Physics
            \item \textbf{TOP : }Thermodynamics, modeling, Process Optimization
        \end{enumerate}
    
        This internship is placed between two divisions (PM and TOP) in a small team composed of : 

        \begin{enumerate}[\hspace{3em}$\bullet$]
            \item \textbf{Noel JAKSE : }Teacher-researcher in the TOP research group, Master and Supervisor of the internship
            \item \textbf{Rafael ESTEVEZ : }Researcher, Co-Supervisor of the internship
            \item \textbf{Thibault MROZ : }Intern Assistant Engineer
        \end{enumerate}

        The TOP Research Group focuses on materials development, thermodynamic phenomena (stability and characterization) and atomistic, thermodynamic, kinetic and reactor modeling. This has applications in the fields of thin films, complex metal alloys and functional materials.

        The PM Research Group focuses on the metallurgy of metals: atomic structure, mechanical and physical properties and oxidation. This has applications in the fields of materials for energy and microelectronics but also for structural materials. 

        The internship is in the field of atomistic modeling and atomic structure. 

\section{Report Outline}
        
