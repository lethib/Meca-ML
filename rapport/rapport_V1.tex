\documentclass[11pt]{report}
\usepackage[utf8]{inputenc}
\usepackage{amsmath}
\usepackage{amsfonts}
\usepackage{amssymb}
\usepackage{fancyhdr}
\usepackage{graphicx}
\usepackage{lipsum}
\usepackage{geometry}
\usepackage{lastpage}
\usepackage{indentfirst}
\usepackage{titlesec}
\usepackage{blindtext, color}
\usepackage{hyperref}
\usepackage{nameref}
\usepackage{tabularx}
\usepackage{glossaries}
\usepackage{xcolor}
\usepackage[toc,page,title,header]{appendix}
\usepackage{enumerate}
\usepackage{caption}
\usepackage[normalem]{ulem}

%-----Couleur des liens cliquables-----%
\definecolor{bleu marine}{HTML}{1E3F5A}
\hypersetup{
    colorlinks=true,
    linkcolor=gray,
    filecolor=magenta,
    urlcolor=bleu marine,
    citecolor = gray,
}
%---------------------------------------%

%-----Mise en forme des légendes-----%
\DeclareCaptionFont{gras}{\bfseries}
\DeclareCaptionLabelFormat{test}{\small\uline{#1~#2}}
\DeclareCaptionTextFormat{test}{\small\uline{#1}}
\DeclareCaptionLabelSeparator{test}{\small\uline{:~}}

\captionsetup{
format=hang,labelformat=test,textformat=test,labelseparator=test, labelfont = gras}
%------------------------------------%

%-----Mise en forme des titres-----%
\definecolor{gray75}{gray}{0.75}
\newcommand{\hsp}{\hspace{20pt}}
\titleformat{\chapter}[hang]{\huge\bfseries}{\thechapter\hsp\textcolor{gray75}{|}\hsp}{0pt}{\bfseries}
\titlespacing*{\chapter}{0pt}{-50pt}{20pt}
\titleformat{\subsection}[hang]{}{\hspace{1em}\bfseries\thesubsection}{1em}{\bfseries}
\titleformat{\subsubsection}[hang]{}{\hspace{2em}\bfseries\thesubsubsection}{1em}{\bfseries}
\renewcommand{\thesubsubsection}{\alph{subsubsection})}
%----------------------------------%

%-----Redéfinition des noms des parties-----%
\renewcommand{\contentsname}{Sommaire}
\renewcommand{\tablename}{Tableau}
\renewcommand{\chaptername}{Chapitre}
\renewcommand\listfigurename{Liste des Figures}
\renewcommand\listtablename{Liste des Tableaux}
\renewcommand{\appendixtocname}{Annexes}
\renewcommand{\appendixpagename}{Annexes}
\renewcommand{\bibname}{Bibliographie}
%-------------------------------------------%

%-----Création nouveau type de colonne-----%
\newcolumntype{M}[1]{>{\centering\arraybackslash}m{#1}}
%------------------------------------------%

%-----Mise en page globale-----%
\graphicspath{{photo/}}
\pagestyle{fancy}

\fancypagestyle{plain}{
    %Chapter appelle le style plain donc obligé de faire ça%
    \fancyhead[L]{\leftmark}
    \fancyhead[R]{\includegraphics[scale = 0.3]{logo phelma.png}}
    \renewcommand\footrulewidth{0.5pt}
    \fancyfoot[C]{\textbf{Page \thepage/\pageref{LastPage}}}
    \setlength{\headsep}{45pt}
}

\fancyhead[L]{\leftmark}
\fancyhead[R]{\includegraphics[scale = 0.3]{logo phelma.png}}
\renewcommand\footrulewidth{0.5pt}
\fancyfoot[C]{\textbf{Page \thepage/\pageref{LastPage}}}
\setlength{\headheight}{46pt}
\setlength{\headsep}{15pt}
\setlength{\topmargin}{-26pt}
\setlength{\textwidth}{15cm}
\setcounter{secnumdepth}{4}
\setcounter{tocdepth}{4}
%-------------------------------%

%-----GLOSSAIRE-----%

%--------------------%


\begin{document}

%-----Première page-----%
\begin{titlepage}
    \hfill
    \begin{center}
        \begin{minipage}{1\linewidth}
        \begin{minipage}{.5\linewidth}
              \begin{flushleft} %aligner à gauche
             \includegraphics[scale=0.5]{logo phelma.png} %% Logo haut gauche
            \end{flushleft}
        \end{minipage}
        \begin{minipage}{.5\linewidth}
              \begin{flushright}    
         %	\includegraphics[scale=.5]{logo2} %% Logo haut droite
            \end{flushright}
        \end{minipage}
        \end{minipage}
        
        \vspace*{\stretch{1}}
        
        \textsc{\LARGE Dialogue entre dynamique moléculaire et milieux continu: définition d'un modèle cohésif fondé sur des informations à l'échelle atomique}

        \bigskip
        \bigskip
        \bigskip
        
        \hrule height 2px \bigskip
        { \huge \bfseries - RAPPORT DE STAGE 2A - \\[0.4cm] }
        
        \hrule height 2px 

        \bigskip
        {\large Année universitaire 2021-2022} \\[0.7cm] 
        
        \begin{figure}[!h]
            \centering
            \includegraphics[scale=1.2]{logo_SIMaP_801.png}
        \end{figure}
              \vfill 
        
        \begin{minipage}[d]{1\linewidth}
            \begin{flushright} \large
            \textbf{Maître de Stage}\\
                \textbf{Noel JAKSE}\\
                0102030405\\
                noel.jakse@grenoble-inp.org
            \end{flushright}
        \end{minipage}
        
        \bigskip
             
        \begin{minipage}[d]{1\linewidth}
              \begin{flushleft} \large
              \textbf{Thibault MROZ}\\
                \textbf{2A - SIM}\\
              \end{flushleft}
         \end{minipage}
          \end{center}        
\end{titlepage}
%-----------------------%

\tableofcontents

\setcounter{chapter}{0}
    \chapter*{Remerciements}\chaptermark{Remerciements}
    \addcontentsline{toc}{chapter}{Remerciements} %car non numéroté%

    % \chapter*{Abstract et Résumé}\chaptermark{Abstract et Résumé}
    % \addcontentsline{toc}{chapter}{Abstract et Résumé}


%\printglossary[title={Glossaire}, toctitle={Glossaire}]%

    %\addcontentsline{toc}{chapter}{Glossaire}%

	\listoffigures
	\addcontentsline{toc}{chapter}{Liste des Figures}

    \listoftables
    \addcontentsline{toc}{chapter}{Liste des Tableaux}

\chapter{Introduction}
    % Blabla introduction 
\section{Contexte et Problématique du Stage} % Expliquer dans quel contexte s'inscrit le stage

    Ce stage s'inscrit dans le cadre d'une analyse multi-échelles de la rupture et plus précisément en menant un diaglogue en dynamique moléculaire et description en milieu continu. Plus précisément, il s'agit d'identifier un modèle de zone cohésive, représentant à l'échelle continu le mécanisme de rupture au travers d'une relation "vecteur-contrainte" - "ouverture". Ce modèle sera identifiable suite à des calculs en dynamique moléculaire qui produisent les expérimentations numériques. \\

    Une partie importante du travail est de mener des simulations de Dynamique Moléculaire sur un cristal de Silicium (Si) pour lequel la rupture a lieu par clivage. Cependant, le Si possédant des propriétés élastiques anisotropes, il est attendu que les propriétés de rupture le soient également. Dès lors, des simulations pour différentes orientations entre plan de la fissure / plans de symétrie critallins seront également à mener. Une approche systématique peut être menée. Néanmoins, la méthodologie associant Machine Learning et Dynamique Moléculaire est à exploiter afin de gagner en temps de calculs. \\
    
    Une fois que le modèle cohésif est identifié, il est ensuite possible d'étudier et de prédire les intercations entre fissuration et microstructure (dans un polycristal par exemple), ainsi qu'entre fissure et cavité. \\

    Il s'agit d'un projet "100\% numérique" et porte un fort intérêt pour les méthodes de simulations ainsi que le Machine Learning.

\chapter{Gestion de Projet}


    \section{Répartition des rôles}


\chapter{Mise en contexte}

    \section{Contexte d'utilisation}


\chapter[Étude d'hydroliennes en milieu artificiel]{Étude des caractéristiques des prototypes d'hydroliennes en milieu artificiel}

    \section{Conception et réalisation du système expérimental}

\chapter[Implantation des hydroliennes sur Grenoble]{Étude du potentiel d'implantation des hydroliennes sur Grenoble}



\chapter*{Conclusion et Perspectives}\chaptermark{Conclusion et Perspectives}
\addcontentsline{toc}{chapter}{Conclusion et Perspectives}


\begin{thebibliography}{9}%Voir sur internet pour le 0 
%Revoir biblio%
    \addcontentsline{toc}{chapter}{Bibliographie}
\end{thebibliography} 

%-----ANNEXES-----%
\begin{appendices}


\fancyhead[L]{ANNEXES}
\fancyhead[R]{\includegraphics[scale = 0.05]{logo phelma.png}}
\renewcommand\footrulewidth{0.5pt}
\fancyfoot[C]{\textbf{Page \thepage/\pageref{LastPage}}}
\setlength{\headheight}{46pt}
\setlength{\headsep}{15pt}
\setlength{\topmargin}{-26pt}
\setlength{\textwidth}{15cm}
\setcounter{secnumdepth}{4}
\setcounter{tocdepth}{4}


\renewcommand{\thesection}{\Alph{section})}

\section{Annexe 1}
\label{section:annexe_1}

\end{appendices}
%--------------------------------------------------%


\end{document}
